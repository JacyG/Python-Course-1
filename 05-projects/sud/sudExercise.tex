\documentclass[smallheadings,12pt]{scrartcl}
\usepackage{marvosym}
\usepackage{color}
\usepackage{float}
\usepackage{geometry}
\usepackage{graphicx}
\usepackage{soul} % underlining text
\geometry{verbose,letterpaper,tmargin=2cm,bmargin=3cm,lmargin=2.5cm,rmargin=3.5cm}
\usepackage[utf8]{inputenc}
\usepackage[bookmarks,colorlinks=true]{hyperref}
\usepackage{listings}

\usepackage[english]{babel}
\usepackage{amsmath}
\usepackage{amsfonts}
\usepackage{amssymb}

%----------------------------------------
\newcommand{\pd}[2]{\frac{\partial #1}{\partial #2}}
\DeclareMathOperator*{\sgn}{sgn}
\DeclareMathOperator*{\sig}{sig}
\DeclareMathOperator*{\argmin}{arg\,min}
\DeclareMathOperator*{\argmax}{arg\,max}

\newcommand{\cmd}[1]{\texttt{#1}}
%--------------------------------------


\begin{document}
\parindent1cm
\pagestyle{myheadings}
\title{Programming a single-user dungeon}
\lstset{language=Python,numbers=left,frame=shadowbox}

\maketitle

\section*{Multi user dungeons}

For a long time, playing role-playing games on the computer meant connecting to a
remote server and interacting via text with the virtual world. These games, called MUDs~(multi user dungeon), were created in the 70s and quickly spread among university students.  
They are also considered the embryo of famous multi player games as Ultima Online, Everquest
and World of Warcraft. A more complete description of its history can be found
on Wikipedia (\url{http://en.wikipedia.org/wiki/MUD}). 

A MUD is a mixture of game and chat, with a text-based interface. This allows a very rich description
of the world, only limited by the designer's imagination. This interface consists of a prompt, where
commands (verbs) are typed and results are output on the screen. There are verbs for moving 
(``\cmd{go [direction]}''), looking around (``\cmd{look}'') and also to act on items of the world. 
For example, ``\cmd{eat apple}'' would apply the verb \cmd{eat} to the object of name \cmd{apple}.
Some of the verbs are actually shortcuts. One can use directions (e.g., \cmd{north}, \cmd{south},
\cmd{east} and \cmd{west}) as verbs. The rooms in this virtual world are connected by directions,
and each has a description of how it looks like.

\section*{Project}

For this project, you will implement a ``single user dungeon'' (SUD). This is the single player version of a MUD. 
The goal is to have a world composed of rooms, each containing a description (how it looks) and linked to other
rooms (to allow moving). A room should also keep track of the items in it. The description of the room and
which items are in it should appear when the command ``look'' is used. Items must have at least a name,
description and respond to verbs. For example, a guitar that is ``used'' (i.e., \cmd{use guitar}) should
play a song. If you look at an item (\cmd{look guitar}), you should see a description of how it looks like. 
You also need to keep track of the player: his name and belongings. 

The simplest way to manage this is to create classes Room, Item and Player. Which variables and
functions should they contain? A suggestion to make your life simpler: define methods for each verb
and call them accordingly. As not every item is the same (some will be edible, others will be playable),
you can create classes derived from Item: Food, Drink, MusicalInstrument, Notebook, etc, which answer to
different verbs, or even behave differently when \cmd{look}'ed. Be creative. You can also play some MUDs
for inspiration (links on Wikipedia).  


\section*{Notes and Hints}
 \begin{enumerate}
 \item Try to start with a simple script which just asks for
 the player's name and allows him to type commands in a prompt. For this, you can loop reading the raw 
 input and split the string in order to get the first word.
 
 \item After you have a working prompt, you can create a single room and have items on it.
 This should allow you to test different verbs.
 
  \item As the player has belongings, you should implement verbs ``pick'' and ``drop'',
  so he can move an item from the world to his inventory and vice-versa. You should also
  create a verb (\cmd{inventory}) to list his items.
 
 \item If everything is working so far, start linking rooms and test movement. Remember that
 if room A is north of room B, it also means that room B is south of room A.
 
 \item Command parsing should understand a lazy player. That is, typing just the letter \cmd{l} should work
 the same as \cmd{look} and the directions can be: \cmd{n}, \cmd{s}, \cmd{e}, \cmd{w}.  
 
 \item How about creating an item that is shy, and screams at the player if it's looked at?
 
 \item Suggestion of worlds: a museum which describes all the Nobel prize winners
 from G\"ottingen; the inner city of G\"ottingen or some fantasy place from a book that you like.
 
 \item Some suggestion of verbs: \cmd{open}, \cmd{press}, \cmd{destroy}, \cmd{kick},
 \cmd{eat}, \cmd{play}, \cmd{read}, \cmd{scribble}, \cmd{paint}. Invent some more and have fun!
 
  \end{enumerate}
 
 
\end{document}
