\documentclass[12pt]{scrartcl}

\usepackage{listings}
\usepackage{hyperref}
\usepackage{color}

\title{Python-exercises\\ Object-oriented programming}

\newcommand{\ind}[1]{_{\mathrm{#1}}}

\begin{document}

%\definecolor{keywords}{RGB}{255,0,90}
%\definecolor{comments}{RGB}{60,179,113}
%\lstset{language=Python,
%  keywordstyle=color{keywords},
%  commentstyle=color{comments}emph}
\definecolor{dkgreen}{rgb}{0,0.6,0}
\definecolor{gray}{rgb}{0.5,0.5,0.5}
\definecolor{mauve}{rgb}{0.58,0,0.82}
 
\lstset{ %
  language=Python,                % the language of the code
  basicstyle=\footnotesize,           % the size of the fonts that are used for the code
  numbers=left,                   % where to put the line-numbers
  numberstyle=\tiny\color{gray},  % the style that is used for the line-numbers
  stepnumber=2,                   % the step between two line-numbers. If it's 1, each line 
                                  % will be numbered
  numbersep=5pt,                  % how far the line-numbers are from the code
  backgroundcolor=\color{white},      % choose the background color. You must add \usepackage{color}
  showspaces=false,               % show spaces adding particular underscores
  showstringspaces=false,         % underline spaces within strings
  showtabs=false,                 % show tabs within strings adding particular underscores
  frame=tRBl,                   % adds a frame around the code
  rulecolor=\color{black},        % if not set, the frame-color may be changed on line-breaks within not-black text (e.g. commens (green here))
  tabsize=2,                      % sets default tabsize to 2 spaces
  captionpos=b,                   % sets the caption-position to bottom
  breaklines=true,                % sets automatic line breaking
  breakatwhitespace=false,        % sets if automatic breaks should only happen at whitespace
  title=\lstname,                   % show the filename of files included with \lstinputlisting;
                                  % also try caption instead of title
  keywordstyle=\color{blue},          % keyword style
  commentstyle=\color{dkgreen},       % comment style
  stringstyle=\color{mauve},         % string literal style
  escapeinside={\%*}{*)},            % if you want to add a comment within your code
  morekeywords={*,...}               % if you want to add more keywords to the set
}


\maketitle

\section{String}
\textit{A simple class}
\begin{enumerate}
\item Write a class with one member variable called \texttt{mystring} and
  initialize it in the constructor.
\item Add a class (static) variable called \texttt{counter} and initialize it
  with 0.
\item Add a method which is given another string \texttt{other} as argument. The purpose of this method shall be to test if \texttt{other} is contained in \texttt{mystring}.\\
\textit{Hint:} Try the \texttt{in} keyword.
\item Change the initializer such that it increments counter by one.
\item Create several instances of this String class and check afterwards how
  many you have.
\item Add a 'destructor' function \texttt{\_\_del\_\_} which decreases
  \texttt{counter} by one. Test it!
\end{enumerate}

\section{Animal}
\textit{Inheritance}\\
\begin{enumerate}
\item Write an \texttt{Animal} base class. Each animal shall have an age and a weight. Once set in the constructor these variables should not be changeable from the outside (you cannot just change the age of an animal as you like, right?). But provide functions that allow to read the values of these variables.\\
\emph{Note:} In Python ``private'' attributes are not really private. But it is the best protection of member attributes against the outside that we can get --- so we take it.
\item Implement a \texttt{speak} and a \texttt{move} method which print an error message (an abstract animal can neither speak nor move).
\item Write a \texttt{Cat} and a \texttt{Fish} class which inherit from \texttt{Animal}. Override the \texttt{speak} and \texttt{move} methods such that they print out an appropriate message when called (something like ``Meow'' for the cat\dots)
\item Test your implementation in the python interpreter by creating instances of \texttt{Cat} and \texttt{Fish}.
\item Try to make sense of what's happening here.
\begin{lstlisting}
> c=Cat(2,3.7)
> c.speak()
Meow.
> Animal.speak(c)
I am an abstract animal and cannot speak.
> num = 3.7
> Animal.speak(num)
[some peculiar Python-error message]
\end{lstlisting}
\item Add an \texttt{eat} method which takes as an argument the object to eat;
  for example: \texttt{mycat.eat(myfish)}. Obviously the fish should be dead
  afterwards, so the method's purpose is to ``kill'' the fish, i.e. the fish
  should not be accessible any more (the \texttt{myfish} instance should be
  deleted). How could you accomplish this?
\end{enumerate}

\section{Neuron}
\textit{A more complex example: using what we've learned so far}\\
The equation determining the membrane potential of a \emph{leaky integrate and fire neuron} is given by 
\[c\ind{M}\dot{V}=-g\ind{L}\left(V-V\ind{L}\right)+i\ind{ext}\]
where $c\ind{M}$ is the membrane potential, $g\ind{L}$ the leak conductance, $V\ind{L}$ the corresponding reversal potential and $i\ind{ext}$ an external current that drives the neuron.

In addition, whenever $V$ becomes larger than a threshold value $V\ind{th}$ a spike is elicited and $V$ is reset to a value $V\ind{r}$.
\begin{enumerate}
\item Write a class which is initialized with the necessary parameters (like $c\ind{M},\dots$, don't forget an initial value for $V$). Except for the external current it should not be possible to change the neuron parameters after instanciation.\\
Cf. the hint in exercise 2.1.
\item Use 
\[ \dot{V}\approx\frac{V(t+h)-V(t)}{h}\]
 to derive an update rule $V(t+h)=\dots$.
\item Implement a method which updates $V$ according to this rule until a time $T$ has passed. In each step append the newly calculated $V$ to a list \texttt{trace}.
\item Now consider that $V$ is reset everytime it crosses the threshold.
\item Add a method that uses ``matplotlib'' to plot the voltage
  trace. As ``matplotlib'' will be treated later, here's a spoiler:
  \begin{lstlisting}
    import matplotlib.pyplot as plt
    plt.ion() # turn on interactive mode
    plt.plot(xvals,yvals)
  \end{lstlisting}
\item Test your neuron for different initial values for $V$ and different
  external currents $i\ind{ext}$.
\item \emph{If you are quick:} Simulate a network of neurons. Each neuron is
  connected to some others (e.g. with a predefined probability). The equation
  to simulate is now 
\[c\ind{M}\dot{V_j}=-g\ind{L}\left(V_j-V\ind{L}\right)+i\ind{ext} + i\ind{j,syn}\]
where 
\[i\ind{j,syn} = \sum\limits_k \sum\limits_{m} w_{jk} K(t-t_{jk}^{(m)})\]
and $K(t)=\delta_{t,0}$. $w_{j,m}$ are weight factors specifying the strength
of the connection between neuron $m$ and $j$ and $t_{jk}^{(m)}$ is the time
when neuron $j$ receives its $m$-th spike from neuron $k$. In other words:
whenever neuron $j$ receives a spike from neuron $k$ its voltage changes by an
amount $w_{jk}$.
  
\end{enumerate}

\section{Zoo}
\textit{Dictionaries, special methods, random numbers, functions as arguments
  to functions}\\
We want to collect some animals (the ones from the \emph{Animal} exercise
above) in a \texttt{Zoo} object. It will consist of a
number of animals and each of them should have a name. 
\begin{enumerate}
\item Provide a way to add new animals to the zoo. 
\item \emph{Special methods.} Implement some special methods like
\begin{itemize}
\item the length-operator \texttt{\_\_len\_\_(self)}, to see how many inhabitants the zoo has. Test with \texttt{len(myzoo)} (where \texttt{myzoo} is an instance of \texttt{Zoo})
\item the comparison operator \texttt{\_\_cmp\_\_(self,rhs)} which should return a negative integer if \texttt{self<rhs}, zero if \texttt{self==rhs}, or a positive integer if \texttt{self>rhs}. Test with \texttt{myzoo1<myzoo2} ...
\item the subscripting operator \texttt{\_\_getitem\_\_(self,name)} to access an animal from the zoo by name (e.g. \texttt{z[``blub''].speak()}),
\item \dots
\end{itemize}

\item The zoo welcomes 1000 new animals. As the administration can't come up
  with so many names at the same time, they shall be named, for the moment, by
  numbers. \\
  Write a method with adds a number $n$ of new animals of different species, age
  and weight. You can use the functions  \texttt{random,randint} and
  \texttt{randrange} from module \texttt{random} to create random numbers.
\item Make it possible to rename animals.
\item For easy book-keeping add a \texttt{select} method which takes one
  argument \texttt{fltr}. \texttt{fltr} is itself a function accepting an
  \texttt{Animal} instance as argument and returning a
  boolean. \texttt{select} return a list of all animals for which the
  \texttt{fltr} function returns \texttt{True}. Test this function:
  \begin{itemize}
  \item select all animals which are younger than 2
  \item select all animals for which age+weight $\leq \pi$
  \item select all \texttt{Cat}s
  \end{itemize}
\item Let the zoo visitor specify a simple (!!!) criterium for animal
  selection. I.\,e. create a method \texttt{visitorSelect} which
  reads(i.e. use \texttt{raw\_input}) a string, \texttt{eval}uate it and then call
  \texttt{select}. 

\end{enumerate}

%\section{Copying and Assignment}
%\textit{Behaving of complex objects under copying and assignment}\\

\section{Temperature}

Write a class that stores a temperature in one unit and allows accessing it in
several other ones
(cf. \url{http://en.wikipedia.org/wiki/Conversion_of_units_of_temperature} for
conversion formulas).\\
\emph{Hint:} Have a look at \texttt{\_\_getattr\_\_, \_\_setattr\_\_ } to
access and set temperatures.

\end{document}
